\section{Results}

We have carried out a variety of simulations which involved different countries and available behaviours, as well as different restrictions and valuations applied by our Kyoto Protocol's backbone systems. An individual simulation, with specific configurations regarding which countries are present, what behaviours are being used, and how services act is referred to as a scenario.

\subsection{Scenarios}

During the design and implementation of our Kyoto Protocol simulator, we made any individual simulation configurable in the following ways. This allows us to make sweeping changes to outcomes and agent activities without altering implementation of components of our system. This allows us to construct theoretical alternative scenarios to the real world Kyoto Protocol.

\begin{description}
\item[Session Length] \hfill \\

Session length refers to the number of years in a given commitment period. The longer the sessions, the longer countries will have to meet their target. If this is set too short, countries will be more likely to miss their targets or cheat. This will have repercussions on their ability to spend and invest. They may also choose to just leave the protocol altogether.

\item[Monitor Price] \hfill \\

This is the price to monitor each country every year. This has been chosen to allow fewer countries than there are in the simulation to be monitored. We were aiming for approximately 10\% of cheating countries to be caught. This was to emulate a more realistic situation where monitoring everyone is not possible. However we didn't want to set the price too high, or cheating might become the optimal behaviour.

\item[Investment in Industry] \hfill \\

This will effect how large an increase in \textsc{gdp} rate countries will see for a given investment. If this is too low, it would take too long for countries to see any benefit, and they would prefer to hang on to their money. Too large, and it would become unrealistically easy to meet targets without trading.

\item[Disabling Monitoring] \hfill \\

This could be achieved by setting the monitor tax to 0. Although our behaviours make little use of cheating, disabling the monitor would yield interesting results in a more realistic simulation. Giving countries free reign over meeting their carbon targets would be a test of the morality in their behaviour (if any). The likely result would be no targets being met and global carbon emissions increasing.

\item[Participants] \hfill \\

Removing the so called `rogue' countries (\textsc{us}) from our simulation would have a large beneficial effect on global emission  targets for the remaining participants. This scenario could be compared to a real-world situation where these countries' outputs don't contribute to the way Kyoto set its targets. Although this probably wouldn't help the global carbon emission problem, countries would be more likely to stay if their targets were easier (see Canada).

On the other hand, we could simulate only having `rogue' countries who behave only according to their own internally-set targets. The resulting behaviour would probably be chaotic and not beneficial to the global emission problem. This would be similar to the possible real-world situation where all countries withdraw their support for Kyoto, but commit to their own reduction targets.

The last hypothetical situation we thought would be of interest was if all countries ratified and were classified as Annex 1. Emission targets would be much simpler to understand (and fairer compared to the \textsc{us}'s current unrattified status). Also no \textsc{cdm} would take place. This would take away one of the easiest ways for a country to meet their targets while keeping their heavy carbon industry. There is much concern in the real world over the effectiveness and morality of using \textsc{cdm} to reduce carbon emissions. Taking away arable land from third world countries that could be used to grow crops, just to plant trees which will take several decades to become effective carbon sinks sounds very dubious.

\item[CO$_2$ Reduction Rate] \hfill \\

The faster countries have to reduce their carbon, the more likely they are to miss their target, cheat and/or leave the protocol. We predict a lower reduction rate would be the most beneficial in the long term.

\item[GDP Investment] \hfill \\

Realistically, will invest a different proportion of \textsc{gdp} into clean initiatives. Due to countries not needing to worry about spending money on anything non-Kyoto related in our system, we set a fixed percentage to be invested in clean initiatives and industry expansion. If this parameter is high, countries will have plenty of money to invest in their own industry and international projects, so targets will be much easier to meet. The opposite would make each move more valuable and although targets would be harder to meet, countries may make `wiser' decisions.

\item[GDP Growth] \hfill \\

We have enforced an upper and lower limit on \textsc{gdp} growth of $\pm$-7\%. This reflects the typical real-world values for the majority of stable countries. Increasing this range would make the market more volatile. During growth, investment in industry will have an exaggerated effect, whereas global recession could make it difficult for anyone to meet their targets.

\item[Market State Factors] \hfill \\

These effect the chances of the global economy being in recession or growth, with a weighting towards remaining stable. These could be varies to emulate the protocol during an extended period of either negative or positive growth. The effects would be very similar to changing the \textsc{gdp} growth range as above.
\end{description}
