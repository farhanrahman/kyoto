\section{Introduction}

At the start of the project, we were introduced to a game consisting of four simple rules:

\begin{enumerate}
\item Each participant is given a number of resources
\item Each participant may put any number of their resources back into a common pool
\item	The resources in the pool are multiplied by a factor N (where $N > 1$)
\item	Each participant may appropriate a given amount
\end{enumerate}

We found that participants did not follow the rules regarding appropriation, and the common pool resource was left abused. This lead to some participants being left with no resources at all. It was therefore decided that more stringent rules were necessary, where participants could be subject to random monitoring. However a certain number of resources from the common pool would need to be used to fund this process. The resulting rules were now as follows:

\begin{enumerate}
	\item Each participant is given a number of resources
	\item Each participant must put at least half of their resources into a common pool
	\item The resources in the pool are multiplied by a factor N (where $N > 1$)
	\item Each participant may appropriate a given amount
	\item Participants may choose to monitor others, at the cost of 1 resource being taken from the common pool.
\end{enumerate}

As predicted, this new rule set lead to a more effective prevention of 'cheating', and thus to a more efficient distribution of the resource. Indeed, this is a good example of the efficacy of common pool resource utilisation when participants act as individuals, and participants agree on rules and monitoring as a collective.

This report details the research, design and implementation of a multi agent simulation of the Kyoto common pool resource system, which in many ways is very similar to the simple game described above.
