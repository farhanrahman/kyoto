\section{Future Developments}

If given more time, there are a number of additional layers of complexity we would like to add to our simulation. 

\subsection{Political Pressure}

A political pressure system would represent the international relations and varying attitudes of countries toward one another considering their performance regarding the Kyoto Protocol. Initially this was a system that we felt should be achievable within the scope of our design. However, as time progressed and the difficulty of other systems within the Kyoto Protocol and country implementations became apparent, we decided that this system, while desirable, was non-essential. Adding a facility for countries to affect others' decisions through a simplified influence modifier would make our simulations more complex and representative of reality.

Political pressure would also allow Non-Annex and rogue states to participate more fully in the Kyoto Protocol even without becoming Annex I members. This would allow the \textsc{us} to exert political pressure on Kyoto member states and encourage them to become rogue states.

\subsection{Global Economy}

As outlined previously in our report, our simulation of the global economy is a vast simplification of reality. We maintained only the level of complexity absolutely necessary for the Kyoto Protocol to be cohesive and for any sanctions and spending patterns to be suitably realistic. If we were to expand the scope of our project, we would aim to implement a more accurate model of the economy. Early on in our project design phase we discussed using real world \textsc{gdp} models to represent the growth of each country over the course of a simulation. However, we decided that countries required feedback in their \textsc{gdp} to affect their behaviour, otherwise it was difficult to model the inverse association between spending on Kyoto Protocol related sustainable projects and promoting economic growth elsewhere in the country.

\subsection{Impoverished Countries}

The behaviour of Non-Annex countries appears to be too successful in simulations.  Although it takes several years, the motion graph shows many Sub-Saharan countries, who are mired poverty, to be extremely successful in improving their \textsc{gdp}. Many external factors affect the wealth of country, none of which are modelled within our simulation.  A simple addition such as factoring a countries Human Development Index when calculating growth may help rebalance a countries propects.
 
\subsection{Industry Modelling}

Our carbon trading model and offset assignment concepts are entirely simplifications of how the Kyoto Protocol responsibilities are distributed in reality. While countries are given specific targets as a part of the Kyoto Protocol, most governments then pass on those mandates to industries operating under their jurisdiction. In reality, the majority of emissions trading is between private companies, rather than governments. If we were to expand the complexity of this component of the Kyoto Protocol, we would create industry agents which operate within countries and alter countries so that they, while still being agents, acted more like monitoring agencies that dictated the activities of their associated industries.

\subsection{Available Country Actions}

Some country behaviours are limited in the way that they can interact with the Kyoto Protocol. Non-Annex countries have a limited set of actions available to them within the scope of the Protocol itself, and Rogue states are even further restricted. The global economy and political pressure systems, as described above, would provide a number of different agents with additional avenues of interaction with the global stage. This could potentially have a large effect of scenario outcomes as agents make differing decisions when confronted with new influencing factors.