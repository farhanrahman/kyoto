\section{Conclusion}

Our Kyoto Protocol simulation engine represents the culmination of weeks of work by a diverse team of software engineers. We have created a modelling system which is vastly interdependent yet highly mutable. Given accurate data there are interesting parallels between our results scenarios and the reality of the last fifteen years of the Kyoto Protocol's existence. The representation of any single value or functionality has been scrutinised by a core team of group coordinators who have overseen the vast majority of the code base throughout the project's implementation.

We have had standout performances in design from some of our individual group members, the wrapping of Presage2 and its associated pieces, the MongoDB, and our own implementation code in an impressive UI which can be accessed remotely from any internet-enabled \textsc{pc}. Intelligent agent implementations have discovered routes to favorable positions in ways that were unforeseen by other group members and even their original designer.

We are aware of limitations to our Kyoto Protocol simulation. Some of our group design decisions led to complicated workaround changes later on in the development cycle and if asked to work on a similar project again, understanding of the underlying framework system, in this case Presage2, and design paradigms, in this case multi-agent systems, would be seen as paramount during the project's inception and planning.

Our models of pricing and other non-core concepts to the Kyoto Protocol are purposeful but still largely arbitrary simplifications of real world factors such as the economy and political influence. These have been explored in more detail in the Future Development section. We also fail to simulate more human factors, such as public opinion and general society's influence on policy.

It is important to note that our Kyoto Protocol system does simulate instances of the Kyoto Protocol that extend further into the future than the real Kyoto Protocol has progressed thus far. Any results beyond this point are conjecture on the part of our artificial intelligence agents, and make the assumptions that world activities can continually be modelled in a way that maps to a current day political climate.

There is a potential for emergent technologies to invalidate any predictions our model makes with regards to future emissions and Kyoto Protocol participation, or even the continued existence of the Protocol itself. Our model does, for the simulated future, demonstrate a generally decreasing global emissions output, even accounting for nations that elect to depart from the Protocol.

The parallels with reality that our realistic scenarios have demonstrated range from general trends to almost eerily specific details of some countries, despite shared artificial intelligence code between the analagous agents. For example, the United States' emissions stabilise a few years after Kyoto is put in place, while China's emissions, unrestricted by the targets and sanctions applied to Annex I countries, continues to grow rapidly in the future, until its carbon emissions eclipsed that of all other Annex I states.

We believe that we have designed and implemented a solid framework that if built upon could provide extremely powerful modelling tools for estimating worldwide \textsc{ghg} emissions. Our systems are suitably interrelated to model the meshed nature of the Kyoto Protocol and its cascading effects on global economies and governmental decisions, while still maintaining a suitably implementable level of complexity.

We also believe that, from a purely multi-agent systems perspective, the Kyoto Protocol is in reality an extremely well tuned \textsc{cpr} implementation which avoids the pitfalls of the most obvious avenues of attacking \CO emissions (e.g. carbon taxation). However, it does have limitations. The lack of penalisation for fraudulent reporting relies heavily upon morality and the incidental difficulty of believably falsifying carbon emissions reports that have been collated from a variety of public and private sources.

Our Kyoto Protocol simulation engine, through countless revisions and cooperative decision making, has become a worthy simulation of \textsc{ghg} emissions and speculative modelling of carbon based economy.

\section{Learning Outcomes}

- working in a group
- coordinating strengths and weaknesses of team members
- scheduling tasks based on dependencies
- balancing workload throughout project so we're not here right now
- difficult to make us all busy all the time
- using ourselves as a common pool resource
- success using github for version control and issue tracking, especially raising issues and assigning them to specific relevant individuals
- difficulty of adapting to a new platform - learning curves, Presage2, minimal java experience, no git experience
- dealing with unfamiliar API
- group structure dissolution
