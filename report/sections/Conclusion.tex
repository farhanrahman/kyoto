\section{Conclusion}

- weaknesses of Kyoto
- what our simulation does well
- general trend of carbon reduction
- interesting parallels with real world for our scenarios
- lack of detail in some areas, can't factor in political pressure stuff / human factors
- fundamental inaccuracies - such as costing
- we are predicting the future
- new technologies in the future

- overall we've designed a solid framework for more powerful expansion
- does encompass a significant amount of relevant complexity

\section{Learning Outcomes}

- working in a group
- coordinating strengths and weaknesses of team members
- scheduling tasks based on dependencies
- balancing workload throughout project so we're not here right now
- difficult to make us all busy all the time
- using ourselves as a common pool resource
- group structure dissolution

Many of the learning outcomes of this project related to working in a group. In particular the challenge with regard to the coordination of member's work in way that is productive. As a group we took an informal approach to leadership with a number of individuals taking lead at different points in the project depending on what most needed to be done. This felt most natural as a group however, in retrospect, we feel that a more formal consistent leadership may have had its benefits. One being making better use of individuals who fell behind with the project.

Similarly, the relatively short time frame of the project meant that to complete our aims we relied on maintaining consistent progress. The dependent nature of components within the project inevitably made this difficult. We noticed this especially toward the end as we brought these components together that there was still a lot of bug fixing and testing left to do.

Early in the project we decided that for ease of management and to divide the tasks we would split into smaller groups. We originally arranged these so that there were four people focused on each behaviour type and the remaining five would focus on the common framework. This structure was soon abandoned (although we still loosely retained our groupings) as we discovered that the common architecture required a greater portion of development. In the end we had roughly one or two people working on each behaviour type with the remaining members working on the common framework (services, protocols, \textsc{ui} etc.). 

- success using github for version control and issue tracking, especially raising issues and assigning them to specific relevant individuals
- difficulty of adapting to a new platform - learning curves, Presage2, minimal java experience, no git experience
- dealing with unfamiliar API

Other learning outcomes related to the tools we were using for the project. We had particular success with the source code management software Git - a tool that is engineered toward snapshot based parallel working. We also relied heavily on Github - a web frontend and git repository host that also provides some incredibly useful functionality such as issue tracking. Github's issue tracking was particularly helpful when we were doing a lot of bug fixing, as testers could open issues easily whenever they found unexpected behaviour which could then be assigned to an individual to be fixed and finally closed when dealt with. Over the course of our project we opened and closed over a 130 issues and submitted around 2500 commits (Git's term for code snapshots).

Many of the JAVA BLAH
