\section{Conclusion}

Our Kyoto Protocol simulation engine represents the culmination of weeks of work by a diverse team of software engineers. We have created a modelling system which is vastly interdependent yet highly mutable. Given accurate data there are interesting parallels between our results scenarios and the reality of the last fifteen years of the Kyoto Protocol's existence. The representation of any single value or functionality has been scrutinised by a core team of group coordinators who have overseen the vast majority of the code base throughout the project's implementation.

We have had standout performances in design from some of our individual group members, the wrapping of Presage2 and its associated pieces, the MongoDB, and our own implementation code in an impressive \textsc{ui} which can be accessed remotely from any internet-enabled \textsc{pc}. Intelligent agent implementations have discovered routes to favorable positions in ways that were unforeseen by other group members and even their original designer.

We are aware of limitations to our Kyoto Protocol simulation. Some of our group design decisions led to complicated workaround changes later on in the development cycle and if asked to work on a similar project again, understanding of the underlying framework system, in this case Presage2, and design paradigms, in this case multi-agent systems, would be seen as paramount during the project's inception and planning.

Our models of pricing and other non-core concepts to the Kyoto Protocol are purposeful but still largely arbitrary simplifications of real world factors such as the economy and political influence. These have been explored in more detail in the Future Development section. We also fail to simulate more human factors, such as public opinion and general society's influence on policy.

It is important to note that our Kyoto Protocol system does simulate instances of the Kyoto Protocol that extend further into the future than the real Kyoto Protocol has progressed thus far. Any results beyond this point are conjecture on the part of our artificial intelligence agents, and make the assumptions that world activities can continually be modelled in a way that maps to a current day political climate.

There is a potential for emergent technologies to invalidate any predictions our model makes with regards to future emissions and Kyoto Protocol participation, or even the continued existence of the Protocol itself. Our model does, for the simulated future, demonstrate a generally decreasing global emissions output, even accounting for nations that elect to depart from the Protocol.

The parallels with reality that our realistic scenarios have demonstrated range from general trends to almost eerily specific details of some countries, despite shared artificial intelligence code between the analagous agents. For example, the United States' emissions stabilise a few years after Kyoto is put in place, while China's emissions, unrestricted by the targets and sanctions applied to Annex I countries, continues to grow rapidly in the future, until its carbon emissions eclipsed that of all other Annex I states.

We believe that we have designed and implemented a solid framework that if built upon could provide powerful modelling tools for estimating worldwide \textsc{ghg} emissions. Our systems are suitably interrelated to model the meshed nature of the Kyoto Protocol and its cascading effects on global economies and governmental decisions, while still maintaining a suitably implementable level of complexity.

We also believe that, from a purely multi-agent systems perspective, the Kyoto Protocol is in reality an extremely well tuned \textsc{cpr} implementation which avoids the pitfalls of the most obvious avenues of attacking \CO emissions (e.g. carbon taxation). However, it does have limitations. The lack of penalisation for fraudulent reporting relies heavily upon morality and the incidental difficulty of believably falsifying carbon emissions reports that have been collated from a variety of public and private sources.

Our Kyoto Protocol simulation engine, through countless revisions and cooperative decision making, has become a worthy simulation of \textsc{ghg} emissions and speculative modelling of carbon based economy.

\section{Learning Outcomes}

Many of the learning outcomes of this project related to working in a group. In particular the challenge coordinating member's work in way that is productive. As a group we took an informal approach to leadership, with a number of individuals taking lead at different points depending on what most needed to be done. This felt most natural as a group, however, in retrospect, we feel that a more formal leadership may have had its benefits. One being better use of individuals who fell behind with the project.

Similarly, the relatively short time frame of the project meant that working concurrently was extremely important. However, the dependent nature of components within the project inevitably made this difficult. This became more prominent as the deadline approached when we brought previously separate components together. The necessity for concurrency ran contrary to the requirement for holistic scenarios that involved code from all contributors.

Early in the project we decided that, for ease of management and to divide the tasks, we would split into smaller groups. We originally arranged so there were four people focused on each behaviour type and the remaining five would focus on the common framework. This structure was soon abandoned (although we still loosely retained our groupings) as we discovered that the common framework required a greater portion of development. In the end we had roughly one or two people working on each behaviour type with the remaining members working on the common framework. (services, protocols, and \textsc{ui} etc.) 

Many of our group members were unfamiliar with one or both of Java and Eclipse. The added complexity of Maven's dependency system on top of Eclipse raised the barrier to entry further. However, adaptation from \textsc{cpp} to Java proved an amenable transition. The facilities provided by a more automatic memory management based language were appreciated and the similar syntax to familiar languages was a great boon.

Other learning outcomes related to the tools we were using for the project. We had particular success with the source code management software Git - a tool that is engineered toward snapshot based parallel development. We also relied heavily on Github - a web frontend and Git repository host that also provides some incredibly useful functionality such as issue tracking. Github's issue tracking was particularly helpful for bug fixing, as testers could open issues easily when they found unexpected behaviour. These could then be assigned to individuals who had experience dealing with the effected sections of code. Over the course of our project we submitted around 2500 commits (Git's term for code snapshots) and closed over 130 issues.
