\section{Future Developments}

If given more time, there are a number of additional layers of complexity we would like to add to our simulation. 

\subsection{Global Economy}



\subsection{Political Pressure}



\subsection{Available Country Actions}

Some country behaviours are limited in the way that they can interact with the Kyoto Protocol. Non-Annex countries have a limited set of actions available to them within the scope of the Protocol itself, and Rogue states are even further restricted. The global economy and political pressure systems, as described above, would provide a number of different agents with additional avenues of interaction with the global stage. This could potentially have a large effect of scenario outcomes as agents make differing decisions when confronted with new influencing factors.

\subsection{Canada}
Canada represents an interesting but not simulated real world use case with regards to the Kyoto Protocol. Canada starts off as an Annex I member. It shared the behaviour of similar Annex I countries pledging to reduce carbon emissions. The economy of Canada is such that it is an active oil producing nation(3rd largest in the world). On top of the oil resources it also has an advanced manufacturing industry which emits a lot of green gases. The dilemma for Canada is that it being one of the youngest countries in north american region as well as among the Annex I countries. Canada also has a considerable amount of arable land area which could be used to its advantage as the 2nd largest country in the world.

Canada's strategy utilises the above features. The first feature is where we check the difference between the carbon emissions for each year against the emission target. This difference is mitigated by choosing the appropriate strategy to reduce emission and at the same time keep it under control. We have the option of choosing between carbon reduction or absorption. This is decided by looking at the cost feasibility of each method. The minimum(most feasible) method is chosen. Also the industry growth and yield is also checked due to the presence of a heavy industry. In the case when we have met our emissions target, the agent invests in industry and increases \textsc{gdp} growth. We also check the \textsc{gdp} growth year by year to make sure we don't starve the economy of any funds which will be invested in carbon reduction. We also check for carbon trade opportunities only if the cost of buying carbon offset is cheaper than the carbon absorption or reduction.

A rather ambitious idea was to check the possibility of a correlation of energy producers but it is only possible by generating randomised numbers for oil companies operating and the number of infrastructure projects they have at present. If the number of projects and energy players are within a certain limit we will be able to reduce emmisions and use a percentage of the available money to spend on carbon absorption schemes.

The country's behaviour decides on a yearly basis whether it's feasible to remain in Kyoto and have positive \textsc{gdp} growth or if leaving Kyoto at the end of the session is more beneficial. If the emissions targets are missed for more than five consecutive years, it is judged counterproductive to remain a member of the Kyoto Protocol. This resembles real world events closely, considering that Canada left Kyoto in 2011.
